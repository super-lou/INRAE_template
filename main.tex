\documentclass{config}
\usepackage[utf8]{inputenc}
\usepackage{blindtext}
\usepackage{float}%H figure table
\usepackage{upgreek}%updelta
\usepackage{indentfirst} %compenser l'absence d'indentation en debut de nouveau par
\usepackage[table]{xcolor} %tableau et couleur dans tableau
\usepackage{hyperref}%lien
\usepackage{wrapfig}
\usepackage{gensymb}%\degree
\usepackage{wasysym}
\definecolor{myGrey}{RGB}{251,251,251}
\usepackage[position=bottom]{subfig}
\usepackage{mathtools}
\usepackage{rotating} % rotation of table
\usepackage{raleway}

% inserting pdf
\usepackage{pdfpages}

% highlight
\usepackage{soul}

% image positionning
\usepackage{graphicx}

% background image
\usepackage{watermark}

% inline 
\usepackage{calc}
\newlength\myheight
\newlength\mydepth
\settototalheight\myheight{Xygp}
\settodepth\mydepth{Xygp}
\setlength\fboxsep{0pt}
\newcommand*\inlinegraphics[1]{%
  \settototalheight\myheight{Xygp}%
  \settodepth\mydepth{Xygp}%
  \raisebox{-\mydepth}{\includegraphics[height=\myheight]{#1}}%
}
\usepackage{lipsum}

% footer
\fancyfoot[L]{
  \watermark{
    \centering
    \put(-3cm, -28.42cm){\includegraphics[width=6.5cm]{cover/AE_light.pdf}}}
  {\hspace*{-1.5cm}\large\bfseries\color{Cyan}\sffamily Titre} \\
  {\hspace*{-1.5cm}\small\normalfont\color{Cyan} Louis HÉRAUT} \\
  {\hspace*{-1.5cm}\small\normalfont\color{LightCyan} INRAE centre Lyon-Grenoble Auvergne-Rhône-Alpes} \\
  {\hspace*{-1.5cm}\small\normalfont\color{LightCyan} Riverly} \\
  {\hspace*{-1.5cm}\small\normalfont\color{LightCyan} février 2023}}


\bibliography{bibliography}


%%%%%%%%%%%%%%%%%%%%%%%%%%%%%%%%%%%%%%%%%%%%%%%%%%%%%%%%%%%%%%%%%%%%%% 
\begin{document}
\thispagestyle{empty}
\thiswatermark{
  \centering
  \put(-3cm, -22.9cm){\includegraphics[width=10.5cm]{cover/AE.pdf}}
  \put(12.76cm, -1.7cm){\includegraphics[width=3.2cm]{cover/INRAE.pdf}}
  \put(0.3cm, -16.46cm){\includegraphics[height=0.7cm]{cover/arrow.pdf}}
}

\vspace*{15cm}

{\noindent\hspace*{1cm} \sffamily\bfseries\color{DarkCyan}\huge Titre}\\

\vspace*{0cm}

{\noindent\hspace*{1cm} \sffamily\bfseries\color{Cyan}\Large Sous-Titre}

\newpage
\watermark{
  \centering
  \put(-3cm, -28.42cm){\includegraphics[width=6.5cm]{cover/AE_light.pdf}}}


%%%%%%%%%%%%%%%%%%%%%%%%%%%%%%%%%%%%%%%%%%%%%%%%%%%%%%%%%%%%%%%%%%%%%% 
\begin{abstract}
  Lorem lipsum
\end{abstract}


%%%%%%%%%%%%%%%%%%%%%%%%%%%%%%%%%%%%%%%%%%%%%%%%%%%%%%%%%%%%%%%%%%%%%% 
\newpage 
{\hypersetup{linkcolor=LightCyan}\tableofcontents}
\newpage


%%%%%%%%%%%%%%%%%%%%%%%%%%%%%%%%%%%%%%%%%%%%%%%%%%%%%%%%%%%%%%%%%%%%%% 
\chapter{Chapter 1}
\section{Section 1}
\begin{wrapfigure}{r}{0.4\textwidth}
  \centering
  \vspace{-4mm}
  \includegraphics[width=0.37\textwidth]{figures/figure1.jpg}
  \vspace{-1mm}
  \captionsetup{width=0.37\textwidth}
  \caption{Une couleur et un style qui change un peu (pas trop quand même)}
  \label{fig:figure1}
  \vspace{-15mm}
\end{wrapfigure}
\lipsum[1][1-4] citation entre parenthèse \citep{foote1856circumstances, doyle1992adventures} et citation intégrée \citet{template}.\par
\lipsum[1][5] \href{https://makaho.sk8.inrae.fr/}{lien}.


%%%%%%%%%%%%%%%%%%%%%%%%%%%%%%%%%%%%%%%%%%%%%%%%%%%%%%%%%%%%%%%%%%%%%% 
\section{Section 2}
\subsection{Sous Section 1}
\lipsum[2]

\begin{itemize}
\item item 1
\item item 2
\end{itemize}
\vspace{2mm}

\lipsum[3]

\begin{table}[H]
  \centering
  {\color{DarkCyan}
    \begin{tabular}{l c c c c}
      \arrayrulecolor{Grey}
      \rowcolor{LightGrey}
      \thead{Variable} & \thead{Compliqué} & \thead{Encore plus compliqué} & \thead{Jsp} \\
      \sethlcolor{Blue}\hl{ } \thead{QJXA} & 10 ans & 3 jours & Aucun \\
      \hline
      \sethlcolor{Blue}\hl{ } \thead{tQJXA} & 10 ans & 3 jours & Aucun \\
      \hline
      \sethlcolor{LightBlue}\hl{ } \thead{tCENfonte} & 10 ans & 3 jours & Aucun \\
      \hline
      \sethlcolor{Green}\hl{ } \thead{QA} & 10 ans & 3 jours & Aucun \\
      \hline
      \sethlcolor{Red}\hl{ } \thead{QMNA} & 10 ans & 3 jours & $1^{\text{ier}}$ Mai / 30 Nov. \\
      \hline
      \sethlcolor{Red}\hl{ } \thead{VCN10} & 10 ans & 3 jours & $1^{\text{ier}}$ Mai / 30 Nov. \\
      \hline 
    \end{tabular}
  }
  \vspace{0.3cm}
  \caption{Tableau récapitulatif des chose compliquées et autres en hydrologie.}
  \label{table:tab}
\end{table}

\lipsum[4]

\subsection{Sous Section 2}
\lipsum[5]

\subsubsection*{Sous sous section 1}
\noindent \lipsum[6][1]

\subsubsection*{Sous sous section 2}
\noindent \noindent \lipsum[6][2]

\subsection{Sous Section 3}
\lipsum[7-8]



%%%%%%%%%%%%%%%%%%%%%%%%%%%%%%%%%%%%%%%%%%%%%%%%%%%%%%%%%%%%%%%%%%%%%% 
\chapter{Chapter 2}
\section{Section 1}
\lipsum[9] {\bfseries\color{Blue}Crue}, {\bfseries\color{LightBlue}Crue Nivale}, {\bfseries\color{Green}Moyennes Eaux} et {\bfseries\color{Red}Étiage}
et parfois des jolies balades en Figure \ref{fig:figure2} (enfin pour ceux qui aime marcher).

\begin{figure}[H]
  \centering
  \vspace{0mm}
  \includegraphics[width=1\textwidth]{figures/figure2.jpg}
  \vspace{-2mm}
  \caption{\href{https://saam.inrae.fr/}{Je l'ai piqué là}.}
  \label{fig:figure2}
  \vspace{-1mm}
\end{figure}

\lipsum[10-11] 


%%%%%%%%%%%%%%%%%%%%%%%%%%%%%%%%%%%%%%%%%%%%%%%%%%%%%%%%%%%%%%%%%%%%%% 
\bibchapter{Bibliographie}
% \section{Bibliographie}
\printbibliography[heading=none]


%%%%%%%%%%%%%%%%%%%%%%%%%%%%%%%%%%%%%%%%%%%%%%%%%%%%%%%%%%%%%%%%%%%%%% 
\newpage
{
  \thiswatermark{}
  \pagestyle{empty}
  \includepdf[pages={1}]{figures/inrae_style_page.pdf}
}


%%%%%%%%%%%%%%%%%%%%%%%%%%%%%%%%%%%%%%%%%%%%%%%%%%%%%%%%%%%%%%%%%%%%%% 
\newpage
{
  \thiswatermark{}
  \pagestyle{empty}
  \includepdf[pages={1}]{figures/inrae_credential_page.pdf}
}


%%%%%%%%%%%%%%%%%%%%%%%%%%%%%%%%%%%%%%%%%%%%%%%%%%%%%%%%%%%%%%%%%%%%%% 
\end{document}


%%% Local Variables:
%%% mode: latex
%%% TeX-master: t
%%% End:
